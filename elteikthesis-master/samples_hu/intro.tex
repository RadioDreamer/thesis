\chapter{Bevezetés}
\label{ch:intro}

%\section{A számítástudomány rövid története}

\section{Membránrendszerek}

A 20. század második felében egyre nagyobb figyelmet kaptak az olyan számítások, amely a természetben vagy valamilyen természeti jelenséghez kapcsolódóan mennek végbe. Ezen megfigyelések eredményeképpen számos új módszert és algoritmust köszönhetünk ennek az időszaknak. Ide tartoznak például a genetikus algoritmusok (általánosabban véve az evolúciós algoritmusok), illetve a neurális hálók. Ezen irányvonal kiterjesztésének tekinthető az 21.század legelején kibontakozó területek mint a membránszámítások és a DNS számítások, amelyet a molekuláris biológia alapjaira modelleztek.

A membránszámítások legnagyobb inspirációját a biológiai sejtek jelentik, amelyek az élet legelemibb építőelemei. A sejtek működésében és struktúrájában pedig elengedhetetlenek a membránok, amelyek a sejten belüli rétegződést határozzák meg. Egy sejtet a környezetétől a legkülső membránja (úgynevezett skin) határol el, a rétegződésért pedig a belső membránok felelősek.  Egy-egy ilyen membrán pedig egy saját belső teret, ún. régiót határol el.
Ezen régiókban pedig kémiai reakciók mehetnek végbe, amely hatására különböző molekulák átvándorolhatnak a membránokon keresztül új régiókba. Ezzel a térbeli struktúráltsággal egészítik ki a membránszámítások a DNS számításokat. Tehát a membránszámítás motivációja, hogy egy olyan modellt tudjuk konstruálni, amely képes a sejtek közötti és sejten belüli molekulák által közvetített információáramlás mechanizmusának leírására, szemléltetésére.

Ezen célkitűzéssel vezette be Gheorge Paun a membránrendszereket 2000-ben. Később az ilyen modelleket P-rendszereknek nevezték el az ő tiszteletére. Ahhoz, hogy a biokémiai folyamatokat modellezni tudja, figyelembe vette ezen folyamatok törvényszerűségeit. Nevezetesen, hogy az ilyen reakciók végbemeneteléhez kellő számban szükség van a reakcióhoz szükséges kezdeti molekulákra és ha ezek rendelkezésre állnak, akkor a reakció biztosan meg is fog történni. Emellett egyszerre több reakció (vagy akár egy reakció többször) is végbemehet, ha van hozzá(juk) elegendő nyersanyag. Speciális esetekben reakció hatására membránok feloldódhatnak, ilyen a benne rejlő molekulák és belső régiók az őt körülvevő régióba kerülnek, illetve az is előfordulhat, hogy a reakciók között valamilyen prioritási sorrend is fennállhat; azaz az egyik reakció nem hajtódhat végre, amíg egy a másik reakciónak lehetősége van rá. Ezen reakciók úgynevezett evolúciós lépésekbe szerveződnek, amelyek során a régiók molekulái és a membránrendszer struktúrája is átalakul.

A számítás központi szereplői az előbbi tulajdonságokkal rendelkező reakciók, amelyek a membránrendszerekben szabályoknak neveznek, a résztvevő molekulákat pedig objektumoknak. Ezek a szabályok régiókhoz vannak rendelve. A szabályok alkalmazására a maximális párhuzamosság elve érvényes, azaz ha a jelenleg alkalmazott szabályok után még alkalmazható lenne egy újabb vagy egy eddig már alkalmazott szabály, akkor az a szabály végre is lesz hajtva abban evolúciós lépésben. Ennek megfelelően a membránrendszer evolúciós lépések sorozatán megy keresztül, egészen addig, amíg már nincsen egyetlen alkalmazható szabály sem. Ilyenkor a számítás eredményét alapértelmezetten a környezetbe kijutó objektumok számának tekintik, de ez egyes rendszerekben másképpen is megadható (például egy kijelölt kimeneti régió objektumainak számával).

Érdeklődésemet a téma iránt az SAT eldöntési probléma aktív P-rendszerekkel (amelyek típust a dolgozatban nem részletezem) való lineáris időben történő megoldása keltette fel. Ez elsőre egy hihetetlen eredmény, hiszen ez volt a legelső probléma, amiről bebizonyították, hogy NP-teljes. Az ellentmondást a membránrendszerek masszív párhuzamosítási képességével lehet megmagyarázni, amelynek segítségével a változók és a klózok számának függvényében igaz, hogy lineáris időben megoldható, viszont ezt a membránrendszert szimuláló Turing gép már nem tudna minden bemeneti példányra lineáris időben leállni. Ezután kezdtem el foglalkozni membránrendszerekkel és úgy éreztem, hogy egy szimulációs szoftver hasznosnak bizonyulna kutatási és oktatási célokra egyaránt.

\section{Megvalósítandó feladat}

A dolgozatom célja egy olyan szoftver elkészítése, amely lehetővé teszi a felhasználó számára különböző felépítésű membránrendszerek megkonstruálását, grafikus módon való megjelenítését, illetve számításainak végrehajtását. A támogatott számítási modellek magukba foglalják a membránrendszerek alapmodelljét (amelyben egyes régiók feloldódhatnak, illetve a szabályok között lehet prioritási sorrend) és a szimport-antiport rendszereket.
A membránrendszer létrehozását a felhasználó egy (megfelelő formátumú) karakterlánc megadásával kezdeményezheti, ami egyszerűen és szemléletesen írja le a membránrendszer struktúráját és opcionálisan tartalmazhatja régionra bontva a benne található objektumokat. A megalkotott membránrendszer ilyenkor még nem rendelkezik evolúciós szabályokkal, azokat egy külön menü segítségével tudja beállítani. Ilyenkor a kiválasztott régió objektumai is szerkeszthetőek.

A szoftver különösen hasznos tud lenni oktatási célokra, hiszen biztosítja a szabályok elemzésének lehetőségét a lépésenkénti futtatás funkcióval, illetve a teljes szimuláció során megkapott eredmények segítségével tanulmányozható, hogy a létrehozott modell a kívánt nyelvet generálja-e, ezzel segítve a helyes és helytelen működés (nem teljeskörű) kiszűrését.
A program biztosít a felhasználó számára egy használati útmutatót, melyben megtalálja a legfontosabb információkat ahhoz, hogy korábbi ismeretek nélkül is képes legyen egy működő membránrendszert megalkotni és számítását szimulálni.

Lehetőség van a membránrendszer aktuális állapotának lementésére, illetve egy korábbi mentés betöltésére. A membránrendszer \textit{JSON} formátumban kerül mentésre, így azt a felhasználónak lehetősége nyílik arra, hogy a fájlban is módosítson, majd a módosított állapotot töltse be. Ezáltal a szoftver biztosítani tudja azt, hogy egy bonyolult vagy gyakran viszsgált membránrendszert nem kell mindig a kezdetektől megalkotni, elegendő egyszer a kívánt konfigurációját előállítani, majd később betölteni a már elmentett verziót.

A feladatot \textit{Python} programozási nyelven, a \textit{Qt} keretrendszer felhasználásával valósítottam meg.