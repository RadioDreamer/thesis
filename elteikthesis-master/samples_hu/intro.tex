\chapter{Bevezetés}
\label{ch:intro}

\section{A számításelmélet rövid története}
A számításelmélet az 1930-as évek középén Alan Turing és Alonzo Church munkássága révén vált külön tudományággá, amelynek központjában a számítások és a számítógépek formális modellezése és elemzése áll \cite{savage2008modelsofcomp}. A terület azóta is kulcsfontosságú szerepet kap az informatikában. Az előbb említett matematikai logikával foglalkozó tudósok és társaik arra a kérdésre szerettek volna választ adni, hogy mit jelent maga a számítás? Ezzel az algoritmus matematikai modelljének megalkotását tűzték ki célul. Természetesen a számításelmélet létrejötte előtt is születtek algoritmusok, de azok nem törekedtek formalitásra, hanem egyszerűen csak rögzítették a végrehajtandó utasítások sorrendjét. A kérdés megválaszolása érdekében Turing bevezette a Turing gépet, Church pedig a lambda kalkulust. Ezen eszközök segítségével elkezdték vizsgálni a kiszámíthatóságot. Azt fedezték fel, hogy vannak olyan számítási feladatok, amelyekre nem adható algoritmikusan válasz, illetve, hogy vannak olyan univerzális számítógépek, amelyek minden másik számításra alkalmas gépet szimulálni tudnak.
Ez vezetett a Church-Turing tézis megfogalmazásához, amely szerint minden olyan matematikailag formalizálható probléma, ami megoldható valamilyen algoritmussal, az Turing géppel is megoldható. Később kiderült, hogy léteznek a Turing géppel azonos számítási erejű modellek, viszont azóta sem találtak ennél nagyobb számítási erővel rendelkező absztrakt modellt. Ezek szerint ha elfogadjuk a tézist, akkor a Turing gép tekinthető az algoritmus formális modelljének.

A tudományág fejlődését ösztönözte és elősegítette a digitális számítógépek megjelenése. A tézis után pár évvel Konrad Zuse elkészítette az első általános célú programozható elektromechanikus számítógépet, amelyet elektromágneses relékkel működtetett. 1945-ben készült el az első teljesen elektromos számítógép, az ENIAC, majd Neumann János 1947-ben megfogalmazta a mai számítógép működésének alapjául szolgáló Neumann-elveket. Azóta a számítógépek számítási kapacitása rohamos ütemben növekedik, ám az akkor fontosnak vélt számítási elemzések és korlátok a mai napig fontos szempontot adnak algoritmusok megalkotásánál.


%Ezek után a terület számos új irányba ágazott el. Az egyik legelső ilyen kérdéskör a programozási nyelvekhez kapcsolódott. A számítógép parancsokkal való ellátása a  kézzel beírt 0 és 1 karakterek sorozata helyett a programozási nyelvek közbeiktatásával történt, amelyek magasabb absztrakciós szintet nyújtottak a programozóknak.

%Az 1950-es években jelentek meg a véges automaták, illetve ekkor hódított teret  a Chomsky által bevezettt nyelvhierarchia, ami a formális nyelvek megfogalmazásában bontakozott ki.

A mai számításelmélet nagyon változazos témák tárházával rendelkezik, 
magába foglalja az algoritmusok, adatszerkezetek, elosztott és párhuzamosított számítások, kvantumszámítások, biológiai számítások vizsgálatát, illetve nem utolsó sorban a bonyolultságelméletet és az információelméletet.
Mindezekből látszik, hogy milyen szerteágazó tematikája és problémaköre van a számításelméletnek és hogy ezek az irányvonalak folyamatosan képesek fejlődni és gazdagodni.

\section{Membránrendszerek}

A 20. század második felében egyre nagyobb figyelmet kaptak az olyan számítások, amely a természetben vagy valamilyen természeti jelenséghez kapcsolódóan mennek végbe \cite{paun2002membrane}. Ezen megfigyelések eredményeképpen megszületett a biológiai számítások területe, amelynek számos új módszert és algoritmust köszönhetünk. Ide tartoznak például a genetikus algoritmusok (általánosabban véve az evolúciós algoritmusok), illetve a neurális hálók. Ezen irányvonal kiterjesztésének tekinthetőek az 21.század legelejére kibontakozó területek mint a membránszámítások és a DNS számítások, amelyek közül az utóbbit a molekuláris biológia alapjaira modelleztek.

A membránszámítások legnagyobb inspirációját a biológiai sejtek jelentik, amelyek az élet legelemibb építőelemei. A sejtek működésében és struktúrájában pedig elengedhetetlenek a membránok, amelyek a sejten belüli rétegződést határozzák meg. Egy sejtet a környezetétől a legkülső membránja (úgynevezett skin) határol el, a rétegződésért pedig a belső membránok felelősek.  Egy-egy ilyen membrán pedig egy saját belső teret, \textit{régiót} határol el.
Ezen régiókban pedig kémiai reakciók mehetnek végbe, amely hatására különböző molekulák átvándorolhatnak a membránokon keresztül új régiókba. Ezzel a térbeli struktúráltsággal egészítik ki a membránszámítások a DNS számításokat.A  membránszámítás motivációja, hogy egy olyan modellt tudjuk konstruálni, amely képes a sejtek közötti és sejten belüli molekulák által közvetített információáramlás mechanizmusának leírására, szemléltetésére.

Ezen célkitűzéssel vezette be Gheorge Paun a membránrendszereket 2000-ben. Később az ilyen modelleket P-rendszereknek is elnevezték az ő tiszteletére. Ahhoz, hogy a biokémiai folyamatokat modellezni tudja, figyelembe vette ezen folyamatok törvényszerűségeit. Nevezetesen, hogy az ilyen reakciók végbemeneteléhez kellő számban szükség van a reakcióhoz szükséges kezdeti molekulákra és ha ezek rendelkezésre állnak, akkor a reakció biztosan meg is fog történni. Emellett egyszerre több reakció (vagy akár egy reakció többször) is végbemehet, ha van hozzá(juk) elegendő nyersanyag. Speciális esetekben egy reakció hatására membránok feloldódhatnak, ilyenkor a benne rejlő molekulák és belső régiók az őt körülvevő régióba kerülnek, illetve az is előfordulhat, hogy a reakciók között valamilyen prioritási sorrend is fennállhat; azaz az egyik reakció nem hajtódhat végre, amíg egy a másik reakciónak lehetősége van rá. Ezen reakciók úgynevezett \textit{evolúciós lépésekbe} szerveződnek, amelyek során a régiók molekulái és a membránrendszer struktúrája is átalakulhat.

A számítás központi szereplői az előbbi tulajdonságokkal rendelkező reakciók, amelyek a membránrendszerekben szabályoknak nevezünk, a résztvevő molekulákat pedig objektumoknak. Ezek a szabályok régiókhoz vannak rendelve. A szabályok alkalmazására a maximális párhuzamosság elve érvényes, azaz ha a jelenleg alkalmazott szabályok után még alkalmazható lenne egy újabb vagy egy eddig már alkalmazott szabály, akkor az a szabály végre is lesz hajtva az adott evolúciós lépésben. Ennek megfelelően a membránrendszer evolúciós lépések sorozatán megy keresztül, egészen addig, amíg már nincsen egyetlen alkalmazható szabály sem. Ilyenkor a számítás eredményét alapértelmezetten a környezetbe kijutó objektumok számának tekinjük, de ez egyes rendszerekben másképpen is megadható (például egy kijelölt kimeneti régió objektumainak számával).

Érdeklődésemet a téma iránt az SAT eldöntési probléma aktív P-rendszerekkel (amely típust a dolgozatban nem részletezem) való lineáris időben történő megoldása keltette fel. Ez elsőre egy hihetetlen eredmény, hiszen ez az egyik legismertebb NP-teljes probléma. Az ellentmondást a membránrendszerek masszív párhuzamosítási képességével lehet megmagyarázni, amelynek segítségével a változók és a klózok számának függvényében igaz, hogy lineáris időben leáll, viszont ezt a membránrendszert szimuláló Turing gép már nem tudna minden bemeneti példányra lineáris időben leállni. Ezután kezdtem el foglalkozni membránrendszerekkel és úgy éreztem, hogy egy szimulációs szoftver hasznosnak bizonyulna kutatási és oktatási célokra egyaránt.

\section{Megvalósítandó feladat}

A dolgozatom célja egy olyan szoftver elkészítése, amely lehetővé teszi a felhasználó számára különböző felépítésű membránrendszerek megkonstruálását, grafikus módon való megjelenítését, illetve számításainak végrehajtását. A támogatott számítási modellek magukba foglalják a membránrendszerek alapmodelljét (amelyben egyes régiók feloldódhatnak, illetve a szabályok között lehet prioritási sorrend) és a szimport-antiport rendszereket.
A membránrendszer létrehozását a felhasználó egy (megfelelő formátumú) karakterlánc megadásával kezdeményezheti, ami egyszerűen és szemléletesen írja le a membránrendszer struktúráját és opcionálisan tartalmazhatja régiokra bontva a benne található objektumokat. A megalkotott membránrendszer ilyenkor még nem rendelkezik evolúciós szabályokkal, azokat egy külön menü segítségével tudja beállítani. Ilyenkor a kiválasztott régió objektumai is szerkeszthetőek.

A szoftver különösen hasznos tud lenni oktatási célokra, hiszen biztosítja a szabályok elemzésének lehetőségét a lépésenkénti futtatás funkcióval, illetve a teljes szimuláció során megkapott eredmények segítségével tanulmányozható, hogy a létrehozott modell a kívánt nyelvet (ebben az esetben a természetes számok valamely részhalmazát) generálja-e, ezzel segítve a helyes és helytelen működés (nem teljeskörű) kiszűrését.
A program biztosít a felhasználó számára egy használati útmutatót, melyben megtalálja a legfontosabb információkat ahhoz, hogy korábbi ismeretek nélkül is képes legyen egy működő membránrendszert megalkotni és számítását szimulálni.

Lehetőség van a membránrendszer aktuális állapotának lementésére, illetve egy korábbi mentés betöltésére. A membránrendszer \textit{JSON} formátumban kerül mentésre, így azt a felhasználónak lehetősége nyílik arra, hogy a fájlban is módosítson, majd a módosított állapotot töltse be. Ezáltal a szoftver biztosítani tudja azt, hogy egy bonyolult vagy gyakran vizsgált membránrendszert nem szükségszerű a kezdetektől megalkotni, elegendő egyszer a kívánt konfigurációját előállítani, majd később betölteni a már elmentett verziót.

A feladatot \textit{Python} programozási nyelven, a \textit{Qt} keretrendszer felhasználásával valósítottam meg.